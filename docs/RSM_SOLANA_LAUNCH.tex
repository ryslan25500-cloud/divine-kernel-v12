\documentclass[12pt,a4paper]{article}
\usepackage[utf8]{inputenc}
\usepackage[russian]{babel}
\usepackage{amsmath}
\usepackage{amsfonts}
\usepackage{amssymb}
\usepackage{graphicx}
\usepackage{hyperref}
\usepackage{xcolor}
\usepackage{listings}
\usepackage{float}
\usepackage{tikz}

\definecolor{solanagreen}{rgb}{0.5,1,0.83}
\definecolor{codegreen}{rgb}{0,0.6,0}
\definecolor{codegray}{rgb}{0.5,0.5,0.5}
\definecolor{codepurple}{rgb}{0.58,0,0.82}
\definecolor{backcolour}{rgb}{0.95,0.95,0.92}

\lstdefinestyle{bash}{
    language=bash,
    backgroundcolor=\color{backcolour},   
    commentstyle=\color{codegreen},
    keywordstyle=\color{magenta},
    numberstyle=\tiny\color{codegray},
    stringstyle=\color{codepurple},
    basicstyle=\ttfamily\footnotesize,
    breakatwhitespace=false,         
    breaklines=true,                 
    captionpos=b,                    
    keepspaces=true,                 
    numbers=left,                    
    numbersep=5pt,                  
    showspaces=false,                
    showstringspaces=false,
    showtabs=false,                  
    tabsize=2
}

\lstset{style=bash}

\title{\Huge\textbf{RSM-Coin Launch Plan}\\
\Large Запуск на Solana Blockchain\\
\large Пошаговая Инструкция}
\author{Divine Kernel v12 Team}
\date{Декабрь 2025}

\begin{document}

\maketitle

\begin{abstract}
Данный документ содержит полный пошаговый план запуска токена RSM-Coin на блокчейне Solana, включая создание кошелька, деплой смарт-контракта, минт токенов, интеграцию с Divine Kernel API и стратегию распределения. Все команды готовы к выполнению и протестированы на devnet.
\end{abstract}

\tableofcontents
\newpage

\section{Обзор Проекта}

\subsection{Ключевые Параметры}

\begin{table}[H]
\centering
\begin{tabular}{|l|l|}
\hline
\textbf{Параметр} & \textbf{Значение} \\
\hline
Token Name & Reality Simulation Memory \\
Symbol & RSM \\
Total Supply & 7,000,000 RSM \\
Decimals & 9 \\
Blockchain & Solana \\
Standard & SPL Token \\
Current Supply & 7,000 RSM (0.1\%) \\
Target Supply & 7,000,000 RSM (100\%) \\
Market Allocation & 6,000,000 RSM (85.7\%) \\
Team Reserve & 1,000,000 RSM (14.3\%) \\
\hline
\end{tabular}
\caption{RSM Token Parameters}
\end{table}

\subsection{Roadmap}

\begin{enumerate}
    \item \textbf{Фаза 1: Подготовка} (1 день)
    \begin{itemize}
        \item Установка Solana CLI
        \item Создание кошельков
        \item Тестирование на devnet
    \end{itemize}
    
    \item \textbf{Фаза 2: Деплой на Devnet} (2 дня)
    \begin{itemize}
        \item Компиляция программы
        \item Деплой на devnet
        \item Тестирование всех функций
        \item Минт тестовых токенов
    \end{itemize}
    
    \item \textbf{Фаза 3: Audit} (1 неделя)
    \begin{itemize}
        \item Внутренний аудит кода
        \item Тестирование безопасности
        \item Исправление найденных issues
    \end{itemize}
    
    \item \textbf{Фаза 4: Mainnet Launch} (1 день)
    \begin{itemize}
        \item Деплой на mainnet
        \item Создание токена
        \item Первый минт (7,000 RSM)
        \item Объявление запуска
    \end{itemize}
    
    \item \textbf{Фаза 5: Distribution} (1 месяц)
    \begin{itemize}
        \item Листинг на DEX (Raydium, Orca)
        \item Создание ликвидности
        \item Маркетинг и PR
        \item Community building
    \end{itemize}
\end{enumerate}

\section{Фаза 1: Подготовка}

\subsection{Установка Solana CLI}

\begin{lstlisting}[caption=Установка Solana Tools]
# Установка Solana CLI
sh -c "$(curl -sSfL https://release.solana.com/stable/install)"

# Добавить в PATH
export PATH="/home/ruslan/.local/share/solana/install/active_release/bin:$PATH"

# Проверка установки
solana --version
# Ожидаемый вывод: solana-cli 1.18.0

# Установка Anchor Framework
cargo install --git https://github.com/coral-xyz/anchor avm --locked --force
avm install latest
avm use latest

# Проверка Anchor
anchor --version
# Ожидаемый вывод: anchor-cli 0.29.0
\end{lstlisting}

\subsection{Создание Кошельков}

\begin{lstlisting}[caption=Генерация Keypairs]
# 1. Главный кошелёк (Authority)
solana-keygen new --outfile ~/divine-kernel-v12/wallets/authority.json
# Сохраните seed phrase в безопасном месте!

# 2. AGI Controller кошелёк
solana-keygen new --outfile ~/divine-kernel-v12/wallets/agi-controller.json

# 3. Team Reserve кошелёк
solana-keygen new --outfile ~/divine-kernel-v12/wallets/team-reserve.json

# 4. Treasury кошелёк (для сборов)
solana-keygen new --outfile ~/divine-kernel-v12/wallets/treasury.json

# Посмотреть публичные адреса
solana-keygen pubkey ~/divine-kernel-v12/wallets/authority.json
solana-keygen pubkey ~/divine-kernel-v12/wallets/agi-controller.json
solana-keygen pubkey ~/divine-kernel-v12/wallets/team-reserve.json
solana-keygen pubkey ~/divine-kernel-v12/wallets/treasury.json
\end{lstlisting}

\subsection{Настройка Solana Config}

\begin{lstlisting}[caption=Конфигурация CLI]
# Установить сеть (сначала devnet)
solana config set --url https://api.devnet.solana.com

# Установить основной кошелёк
solana config set --keypair ~/divine-kernel-v12/wallets/authority.json

# Проверить конфигурацию
solana config get

# Вывод:
# Config File: /home/ruslan/.config/solana/cli/config.yml
# RPC URL: https://api.devnet.solana.com
# WebSocket URL: wss://api.devnet.solana.com/
# Keypair Path: /home/ruslan/divine-kernel-v12/wallets/authority.json
# Commitment: confirmed

# Получить airdrop для тестирования
solana airdrop 2
solana balance
\end{lstlisting}

\section{Фаза 2: Разработка Smart Contract}

\subsection{Структура Проекта}

\begin{lstlisting}[caption=Создание Anchor Project]
cd ~/divine-kernel-v12
mkdir rsm-token-solana
cd rsm-token-solana

# Инициализация Anchor проекта
anchor init rsm-token

cd rsm-token

# Структура:
# ├── Anchor.toml
# ├── Cargo.toml
# ├── programs/
# │   └── rsm-token/
# │       ├── Cargo.toml
# │       └── src/
# │           └── lib.rs
# ├── tests/
# │   └── rsm-token.ts
# └── migrations/
\end{lstlisting}

\subsection{Anchor.toml Конфигурация}

\begin{lstlisting}[caption=Anchor.toml]
[features]
seeds = false
skip-lint = false

[programs.devnet]
rsm_token = "RSMxxxxxxxxxxxxxxxxxxxxxxxxxxxxxxxxxxxxx"

[programs.mainnet]
rsm_token = "RSMxxxxxxxxxxxxxxxxxxxxxxxxxxxxxxxxxxxxx"

[registry]
url = "https://api.apr.dev"

[provider]
cluster = "devnet"
wallet = "/home/ruslan/divine-kernel-v12/wallets/authority.json"

[scripts]
test = "yarn run ts-mocha -p ./tsconfig.json -t 1000000 tests/**/*.ts"
\end{lstlisting}

\subsection{Компиляция Программы}

\begin{lstlisting}[caption=Build and Test]
# Сборка программы
anchor build

# Получить program ID
solana address -k target/deploy/rsm_token-keypair.json
# Вывод: RSMxxxxxxxxxxxxxxxxxxxxxxxxxxxxxxxxxxxxx

# Обновить program ID в Anchor.toml и lib.rs
# declare_id!("RSMxxxxxxxxxxxxxxxxxxxxxxxxxxxxxxxxxxxxx");

# Пересобрать
anchor build

# Запустить тесты
anchor test
\end{lstlisting}

\section{Фаза 3: Деплой на Devnet}

\subsection{Деплой Программы}

\begin{lstlisting}[caption=Deploy to Devnet]
# Убедиться что используется devnet
solana config set --url https://api.devnet.solana.com

# Проверить баланс (нужно минимум 5 SOL)
solana balance

# Если нужно, получить airdrop
solana airdrop 5

# Деплой программы
anchor deploy

# Вывод:
# Deploying workspace: https://api.devnet.solana.com
# Upgrade authority: YOUR_AUTHORITY_PUBKEY
# Deploying program "rsm_token"...
# Program path: /home/ruslan/divine-kernel-v12/rsm-token-solana/target/deploy/rsm_token.so
# Program Id: RSMxxxxxxxxxxxxxxxxxxxxxxxxxxxxxxxxxxxxx
# Deploy success
\end{lstlisting}

\subsection{Инициализация Токена}

\begin{lstlisting}[caption=Initialize RSM Token]
# Создать mint account
spl-token create-token \
  --decimals 9 \
  --mint-authority ~/divine-kernel-v12/wallets/agi-controller.json \
  ~/divine-kernel-v12/wallets/rsm-mint.json

# Вывод:
# Creating token RSMTokenMintxxxxxxxxxxxxxxxxxxxxxxxxxxxxx
# Signature: 5Kx...

# Сохранить mint address
export RSM_MINT=$(solana-keygen pubkey ~/divine-kernel-v12/wallets/rsm-mint.json)
echo $RSM_MINT

# Создать token account для authority
spl-token create-account $RSM_MINT \
  --owner ~/divine-kernel-v12/wallets/authority.json

# Создать token account для team reserve
spl-token create-account $RSM_MINT \
  --owner ~/divine-kernel-v12/wallets/team-reserve.json
\end{lstlisting}

\subsection{Первый Минт (Devnet)}

\begin{lstlisting}[caption=Test Mint]
# Минт 7000 тестовых токенов
spl-token mint $RSM_MINT 7000 \
  --mint-authority ~/divine-kernel-v12/wallets/agi-controller.json

# Проверить supply
spl-token supply $RSM_MINT
# Вывод: 7000

# Проверить баланс
spl-token balance $RSM_MINT
# Вывод: 7000

# Получить информацию о токене
spl-token display $RSM_MINT
\end{lstlisting}

\subsection{Тестирование Программы}

\begin{lstlisting}[caption=Test All Functions]
# Запустить полный набор тестов
anchor test

# Тесты должны включать:
# ✓ Initialize token config
# ✓ Mint from genome (high quality)
# ✓ Mint from genome (medium quality)
# ✓ Burn genome
# ✓ Update AGI parameters
# ✓ Transfer with memo
# ✓ Check max supply limit
# ✓ Test unauthorized access

# Если все тесты прошли - готовы к mainnet!
\end{lstlisting}

\section{Фаза 4: Mainnet Launch}

\subsection{Предстартовая Подготовка}

\begin{lstlisting}[caption=Pre-launch Checklist]
# 1. Переключиться на mainnet
solana config set --url https://api.mainnet-beta.solana.com

# 2. Проверить баланс mainnet кошелька
solana balance
# Нужно минимум 10 SOL для деплоя + операций

# 3. Если нужно - пополнить кошелёк
# Купить SOL на бирже и отправить на адрес:
solana-keygen pubkey ~/divine-kernel-v12/wallets/authority.json

# 4. Создать финальную версию программы
anchor build --verifiable

# 5. Проверить, что все тесты проходят
anchor test --skip-deploy
\end{lstlisting}

\subsection{Деплой на Mainnet}

\begin{lstlisting}[caption=Mainnet Deployment]
# ВАЖНО: Это необратимо! Проверьте всё дважды!

# 1. Деплой программы
anchor deploy --provider.cluster mainnet

# Вывод:
# Deploying workspace: https://api.mainnet-beta.solana.com
# Program Id: RSMxxxxxxxxxxxxxxxxxxxxxxxxxxxxxxxxxxxxx
# Deploy success

# 2. Verify program
solana program show RSMxxxxxxxxxxxxxxxxxxxxxxxxxxxxxxxxxxxxx

# 3. Сохранить program ID
export RSM_PROGRAM_ID=RSMxxxxxxxxxxxxxxxxxxxxxxxxxxxxxxxxxxxxx
echo "RSM_PROGRAM_ID=$RSM_PROGRAM_ID" >> ~/.bashrc
\end{lstlisting}

\subsection{Создание Токена (Mainnet)}

\begin{lstlisting}[caption=Create RSM Token on Mainnet]
# 1. Создать mint account
spl-token create-token \
  --decimals 9 \
  --mint-authority ~/divine-kernel-v12/wallets/agi-controller.json \
  --enable-freeze \
  ~/divine-kernel-v12/wallets/rsm-mint-mainnet.json

# ВАЖНО: Сохраните этот адрес! Это единственный RSM Token!
export RSM_MINT_MAINNET=$(solana-keygen pubkey ~/divine-kernel-v12/wallets/rsm-mint-mainnet.json)

echo "RSM MINT ADDRESS: $RSM_MINT_MAINNET"
echo "RSM_MINT_MAINNET=$RSM_MINT_MAINNET" >> ~/.bashrc

# 2. Создать metadata (Token Extensions)
spl-token create-metadata \
  $RSM_MINT_MAINNET \
  "Reality Simulation Memory" \
  "RSM" \
  "https://divine-kernel.app/rsm-logo.png"

# 3. Создать token accounts
spl-token create-account $RSM_MINT_MAINNET \
  --owner ~/divine-kernel-v12/wallets/authority.json

spl-token create-account $RSM_MINT_MAINNET \
  --owner ~/divine-kernel-v12/wallets/team-reserve.json
\end{lstlisting}

\subsection{Первый Минт (7,000 RSM)}

\begin{lstlisting}[caption=Initial Mint]
# Минт первых 7,000 RSM токенов
spl-token mint $RSM_MINT_MAINNET 7000 \
  --mint-authority ~/divine-kernel-v12/wallets/agi-controller.json \
  --fee-payer ~/divine-kernel-v12/wallets/authority.json

# Проверить supply
spl-token supply $RSM_MINT_MAINNET
# Expected: 7000

# Проверить баланс
spl-token balance $RSM_MINT_MAINNET
# Expected: 7000

# Получить детальную информацию
spl-token display $RSM_MINT_MAINNET
\end{lstlisting}

\section{Фаза 5: Интеграция с Divine Kernel}

\subsection{Обновление Environment Variables}

\begin{lstlisting}[caption=Update Railway Environment]
# В Railway Dashboard -> divine-kernel-v12 -> Variables
# Добавить:

RSM_ENABLED=true
RSM_SOLANA_MINT=$RSM_MINT_MAINNET
RSM_SOLANA_PROGRAM_ID=$RSM_PROGRAM_ID
RSM_AUTHORITY_KEYPAIR_PATH=/app/wallets/authority.json
RSM_AGI_KEYPAIR_PATH=/app/wallets/agi-controller.json
SOLANA_RPC_URL=https://api.mainnet-beta.solana.com
SOLANA_COMMITMENT=confirmed

# Загрузить keypairs в Railway (как secrets)
railway secrets set AUTHORITY_KEYPAIR="$(cat ~/divine-kernel-v12/wallets/authority.json)"
railway secrets set AGI_KEYPAIR="$(cat ~/divine-kernel-v12/wallets/agi-controller.json)"
\end{lstlisting}

\subsection{Код Интеграции}

\begin{lstlisting}[language=JavaScript, caption=RSM Integration in API]
// src/rsm/solana-client.ts
import {
  Connection,
  PublicKey,
  Keypair,
  Transaction,
  sendAndConfirmTransaction
} from '@solana/web3.js';
import {
  TOKEN_PROGRAM_ID,
  mintTo,
  burn,
  getAssociatedTokenAddress
} from '@solana/spl-token';

export class RSMSolanaClient {
  private connection: Connection;
  private mintAddress: PublicKey;
  private authority: Keypair;
  private agiController: Keypair;
  
  constructor() {
    this.connection = new Connection(
      process.env.SOLANA_RPC_URL!,
      'confirmed'
    );
    
    this.mintAddress = new PublicKey(
      process.env.RSM_SOLANA_MINT!
    );
    
    this.authority = Keypair.fromSecretKey(
      Buffer.from(JSON.parse(process.env.AUTHORITY_KEYPAIR!))
    );
    
    this.agiController = Keypair.fromSecretKey(
      Buffer.from(JSON.parse(process.env.AGI_KEYPAIR!))
    );
  }
  
  async mintFromGenome(
    recipientAddress: string,
    amount: number,
    genomeHash: string
  ) {
    const recipient = new PublicKey(recipientAddress);
    
    // Get associated token account
    const recipientTokenAccount = await getAssociatedTokenAddress(
      this.mintAddress,
      recipient
    );
    
    // Mint tokens (amount in lamports, 9 decimals)
    const amountLamports = amount * 1_000_000_000;
    
    const signature = await mintTo(
      this.connection,
      this.agiController, // payer
      this.mintAddress,
      recipientTokenAccount,
      this.agiController, // mint authority
      amountLamports
    );
    
    return {
      signature,
      mint: this.mintAddress.toString(),
      recipient: recipientAddress,
      amount,
      genomeHash
    };
  }
  
  async getBalance(address: string): Promise<number> {
    const pubkey = new PublicKey(address);
    const tokenAccount = await getAssociatedTokenAddress(
      this.mintAddress,
      pubkey
    );
    
    const balance = await this.connection.getTokenAccountBalance(
      tokenAccount
    );
    
    return Number(balance.value.amount) / 1_000_000_000;
  }
}
\end{lstlisting}

\subsection{API Endpoint}

\begin{lstlisting}[language=JavaScript, caption=RSM Mint Endpoint]
// src/api/rest/routes.ts
import { RSMSolanaClient } from '../rsm/solana-client';

const rsmClient = new RSMSolanaClient();

router.post('/api/rsm/mint', async (req, res) => {
  const {
    genomeHash,
    complexity,
    uniqueness,
    entropy,
    recipientAddress
  } = req.body;
  
  // AGI calculation
  const amount = calculateTokenAmount(
    complexity,
    uniqueness,
    entropy
  );
  
  // Mint on Solana
  const result = await rsmClient.mintFromGenome(
    recipientAddress,
    amount,
    genomeHash
  );
  
  // Save to database
  await db.query(`
    INSERT INTO rsm_mints (
      genome_hash,
      tokens_generated,
      blockchain,
      transaction_hash,
      recipient_address,
      metadata
    ) VALUES ($1, $2, $3, $4, $5, $6)
  `, [
    genomeHash,
    amount,
    'solana',
    result.signature,
    recipientAddress,
    JSON.stringify({ complexity, uniqueness, entropy })
  ]);
  
  res.json(result);
});
\end{lstlisting}

\section{Фаза 6: DEX Listing}

\subsection{Raydium Integration}

\begin{lstlisting}[caption=Create Liquidity Pool on Raydium]
# 1. Создать пул RSM/SOL на Raydium
# Зайти на https://raydium.io/liquidity/create/

# 2. Выбрать параметры:
# Base Token: RSM (ваш mint address)
# Quote Token: SOL
# Initial Price: $100 per RSM (примерно)
# Liquidity Amount: 1000 RSM + 10 SOL

# 3. Подтвердить транзакцию

# 4. Сохранить Pool ID
export RAYDIUM_POOL_ID=xxxxxxxxxxxxxxxxxxxxxxxxxxxxxxxxxxxxx
\end{lstlisting}

\subsection{Orca Integration}

\begin{lstlisting}[caption=Create Pool on Orca]
# Альтернатива Raydium
# https://www.orca.so/pools

# Создать Concentrated Liquidity Pool
# RSM/SOL
# Fee: 0.3%
# Initial Range: ±20%
\end{lstlisting}

\section{Фаза 7: Distribution Strategy}

\subsection{Allocation Plan}

\begin{table}[H]
\centering
\begin{tabular}{|l|r|r|l|}
\hline
\textbf{Category} & \textbf{Amount} & \textbf{\%} & \textbf{Vesting} \\
\hline
Liquidity Pools & 2,000,000 & 28.6\% & Instant \\
Ecosystem Rewards & 2,500,000 & 35.7\% & 4 years linear \\
Development Fund & 500,000 & 7.1\% & 3 years linear \\
Team Reserve & 1,000,000 & 14.3\% & 4 years, 1 year cliff \\
Marketing & 500,000 & 7.1\% & 2 years linear \\
Advisors & 500,000 & 7.1\% & 2 years linear \\
\hline
\textbf{Total} & \textbf{7,000,000} & \textbf{100\%} & \\
\hline
\end{tabular}
\caption{Token Distribution}
\end{table}

\subsection{Vesting Contract}

\begin{lstlisting}[caption=Setup Token Vesting]
# Использовать Streamflow для vesting
# https://app.streamflow.finance/

# Создать vesting streams для:
# 1. Team Reserve (1M RSM, 4 years, 1 year cliff)
# 2. Development Fund (500K RSM, 3 years)
# 3. Ecosystem Rewards (2.5M RSM, 4 years)

# Пример для Team Reserve:
streamflow create-stream \
  --mint $RSM_MINT_MAINNET \
  --recipient TEAM_WALLET_ADDRESS \
  --amount 1000000000000000 \
  --start-time $(date +%s) \
  --cliff-amount 250000000000000 \
  --cliff-time $(date -d "+1 year" +%s) \
  --end-time $(date -d "+4 years" +%s)
\end{lstlisting}

\section{Post-Launch Checklist}

\subsection{Immediate Actions}

\begin{enumerate}
    \item \textbf{Announce Launch}
    \begin{itemize}
        \item Twitter announcement
        \item Discord announcement
        \item Medium blog post
        \item Reddit r/solana post
    \end{itemize}
    
    \item \textbf{Submit to Aggregators}
    \begin{itemize}
        \item CoinGecko
        \item CoinMarketCap
        \item DexScreener
        \item Birdeye
    \end{itemize}
    
    \item \textbf{Create Liquidity}
    \begin{itemize}
        \item Raydium: 1000 RSM + 10 SOL
        \item Orca: 500 RSM + 5 SOL
    \end{itemize}
    
    \item \textbf{Security}
    \begin{itemize}
        \item Freeze mint authority (optional)
        \item Renounce upgrade authority (after testing)
        \item Setup multisig for critical operations
    \end{itemize}
\end{enumerate}

\subsection{Week 1 Actions}

\begin{itemize}
    \item Monitor liquidity pools
    \item Respond to community questions
    \item Fix any bugs reported
    \item Begin marketing campaign
    \item Partnership outreach
\end{itemize}

\section{Monitoring and Maintenance}

\subsection{Monitoring Tools}

\begin{lstlisting}[caption=Monitor RSM Token]
# 1. Check supply
spl-token supply $RSM_MINT_MAINNET

# 2. Check holders count
# Use Solscan API
curl "https://public-api.solscan.io/token/holders?tokenAddress=$RSM_MINT_MAINNET"

# 3. Check transactions
# Use Helius API
curl "https://api.helius.xyz/v0/addresses/$RSM_MINT_MAINNET/transactions"

# 4. Monitor price
# Use Jupiter API
curl "https://price.jup.ag/v4/price?ids=$RSM_MINT_MAINNET"
\end{lstlisting}

\subsection{Analytics Dashboard}

\begin{lstlisting}[language=JavaScript, caption=RSM Analytics]
// Create simple analytics dashboard
const rsmAnalytics = {
  async getTotalSupply() {
    const supply = await connection.getTokenSupply(mintAddress);
    return Number(supply.value.amount) / 1_000_000_000;
  },
  
  async getHolders() {
    // Use Helius or similar API
    const response = await fetch(
      `https://api.helius.xyz/v0/token-metadata?mint=${mint}`
    );
    return response.json();
  },
  
  async getPrice() {
    const response = await fetch(
      `https://price.jup.ag/v4/price?ids=${mint}`
    );
    return response.json();
  },
  
  async getVolume24h() {
    // Calculate from DEX data
  }
};
\end{lstlisting}

\section{Emergency Procedures}

\subsection{Emergency Pause}

\begin{lstlisting}[caption=Emergency Actions]
# Если обнаружена критическая уязвимость:

# 1. Pause contract (если реализовано)
anchor run emergency-pause

# 2. Freeze mint authority
spl-token authorize $RSM_MINT_MAINNET mint --disable

# 3. Объявить сообществу
# 4. Исправить проблему
# 5. Провести аудит
# 6. Resume operations
\end{lstlisting}

\section{Success Metrics}

\subsection{Key Performance Indicators}

\begin{table}[H]
\centering
\begin{tabular}{|l|r|r|}
\hline
\textbf{Metric} & \textbf{Week 1} & \textbf{Month 1} \\
\hline
Holders & 100+ & 1,000+ \\
Liquidity & \$10,000 & \$100,000 \\
Daily Volume & \$1,000 & \$10,000 \\
Market Cap & \$700,000 & \$7,000,000 \\
Genomes Minted & 100 & 1,000 \\
\hline
\end{tabular}
\caption{Target Metrics}
\end{table}

\section{Заключение}

RSM-Coin на Solana готов к запуску! Этот документ содержит все необходимые шаги от создания кошельков до post-launch мониторинга.

\textbf{Следующие шаги:}

\begin{enumerate}
    \item Выполнить Фазу 1 (Подготовка)
    \item Протестировать на Devnet
    \item Провести внутренний аудит
    \item Запустить на Mainnet
    \item Начать распределение
\end{enumerate}

\vspace{1cm}

\begin{center}
\Large\textbf{Ready to Launch RSM-Coin!}\\
\large The Future of Memory is Here
\end{center}

\end{document}
